\documentclass{article}
\usepackage{graphicx}
\usepackage[utf8]{inputenc}

\begin{document}

\title{Métodos Computacionales - Tarea 5}
\author{David Barbosa}

\maketitle


\section{Introduction}
Este estudio consiste en medir las velocidades de las estrellas en funcion de la distancia de estas al centro de la galaxia y luego, a partir de estas velocidades uno puede inferir la cantidad de masa que debe tener la galaxia

\begin{equation}
    \label{simple_equation}
    \alpha = \sqrt{ \beta }
\end{equation}

\subsection{Ecuacion modelada}




\begin{figure}
    \centering
    \includegraphics[scale=1]{v_exp_mod.png}
    \caption{Datos experimentales en azul y modelo en línea Naranja}
    \label{fig:exp_vs_mod}
\end{figure}

El modelo:
$$ V_{circ} = \frac{\sqrt{M_b}R}{(R^2+b_b^2)^{3/4}} + 
\frac{\sqrt{M_d}R}{(R^2+(b_d+a_d)^2)^{3/4}} +
\frac{\sqrt{M_h}}{(R^2+a_h^2)^{1/4}} $$

\section{Conclusion}
Vemos una evidente relacion directa o una tendencia similar entre la curva generada por el modelo y el comportamiento de los datos experimentales, es decir los datos obtenidos a partir de la informacion disponible.

\end{document}